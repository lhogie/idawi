\documentclass[11pt]{beamer}
\usetheme{JuanLesPins}
\usepackage[utf8]{inputenc}
\usepackage[english]{babel}
\usepackage{amsmath}
\usepackage{listings, xcolor}
\usepackage{amsfonts}
\usepackage{amssymb}
\usepackage{graphicx}
\author{Luc Hogie}
\title{Conception and implementation of a distributed platform for the experimentation of distributed computing in the IOT}
%\setbeamercovered{transparent} 
%\setbeamertemplate{navigation symbols}{} 
%\logo{} 
\institute{Cnrs/Inria/Université Côte d'Azur} 
%\date{} 
%\subject{} 
\begin{document}

\begin{frame}
\titlepage
\end{frame}

%\begin{frame}
%\tableofcontents
%\end{frame}


\begin{frame}
\frametitle{Conception and implementation of a distributed platform for the experimentation of distributed computing in the IOT}
JThings defines a P2P network of communicating components.
It will be used at Université Côte d'Azur/Inria/I3S as soon as it is ready, to:
\begin{itemize}
	\item investigate decentralized algorithms for IOT
	\item provides distributed DB for time-based scientific data
\end{itemize}
The main idea behind JThings is to be able to:
\begin{itemize}
	\item deploy used-defined components on computers
	\item expose a complete yet simple communication API
	\item efficiently execute parallel/distributed code
\end{itemize}
\end{frame}


\begin{frame}
\frametitle{\#1}
The student will have to:
\begin{itemize}
	\item implement the following typical use cases
		\begin{itemize}
			\item distributed/parallel computation on Inria cluster
			\item IOT network simulation
		\end{itemize}
	\item identify flaws and limitations
	\item propose/implement solutions and related unit tests
\end{itemize}
\end{frame}


\begin{frame}
\frametitle{\#2 --- interoperability using REST}
The student will have to:
\begin{itemize}
	\item understand the architecture of  Things' REST interface
	\item understand the requirements of Grafana's REST interface
	\item adapt JThings to be able to connect both
	\item defining Grafana workbenches to monitor JThings
\end{itemize}
\end{frame}


\begin{frame}
\frametitle{\#3 ---  Web monitoring interface}
The student will have to:
\begin{itemize}
	\item make a State of the Art of Web libraries for interactive data visualization
	\item identify the changes in JThings in order to enable interoperability (that I will implement)
	\item implement a Web-based demonstrator (most probably in JavaScript)
\end{itemize}
\end{frame}


\begin{frame}
\frametitle{Working conditions}
Depending of the sanitary situation:
\begin{description}
	\item[on site] the student would have an office at Inria
	\item[teleworking] we would maintain a permanent contact using a collaborative solution (now using  Slack with a student and it's just fine).
\end{description}
\end{frame}






\end{document}
